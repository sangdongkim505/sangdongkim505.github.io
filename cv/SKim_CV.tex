%%%%%%%%%%%%%%%%%%%%%%%%%%%%%%%%%%%%%%%%%
% Important note:
% This template requires the resume.cls file to be in the same directory as the
% .tex file. The resume.cls file provides the resume style used for structuring the
% document.
%
%%%%%%%%%%%%%%%%%%%%%%%%%%%%%%%%%%%%%%%%%

%----------------------------------------------------------------------------------------
%	PACKAGES AND OTHER DOCUMENT CONFIGURATIONS
%----------------------------------------------------------------------------------------

\documentclass{cv} % Use the custom resume.cls style
\usepackage{mathpazo}
\usepackage{hyperref}
\usepackage{adjustbox}
\usepackage{setspace}
\setstretch{1.15}
\usepackage[left=0.75in,top=0.6in,right=0.75in,bottom=0.6in]{geometry} % Document margins
\newcommand{\tab}[1]{\hspace{.2667\textwidth}\rlap{#1}}
\newcommand{\itab}[1]{\hspace{0em}\rlap{#1}}
\name{Sangdong Kim} % Your name

\begin{document}

%----------------------------------------------------------------------------------------
%	NAME AND CONTACT INFORMATION SECTION
%----------------------------------------------------------------------------------------
\begin{minipage}{0.49\textwidth}
	\begin{flushleft}
		\textbf{Office Address} \\
		Department of Economics \\
		410 Arps Hall \\
		1945 N. High St.  \\
		Columbus, OH 43210
	\end{flushleft}
\end{minipage}
\begin{minipage}{0.49\textwidth}
	\begin{flushright}
		\textbf{Contact} \\
		\href{https://sangdongkim.com}{https://sangdongkim.com}\\
		\href{mailto:kim.7984@osu.edu}{kim.7984@osu.edu} \\
		Last updated: \today
		\\
	\end{flushright}
\end{minipage}
\bigskip

%----------------------------------------------------------------------------------------
%	EDUCATION SECTION
%----------------------------------------------------------------------------------------
\begin{rSection}{Education}

%--copy and paste this region  if you need more--
{Ph.D. in Economics, The Ohio State University} \hfill {\em 2026 (expected)} \\
	% \begin{adjustbox}{minipage=0.95\textwidth, right}
	% 	{Committee: Gabriel Mihalache (advisor), Sean McCrary, Julia K. Thomas}
	% \end{adjustbox} \\
{M.A. in Economics, Seoul National University, South Korea} \hfill {\em 2020} \\
{B.A. in Economics, College of Liberal Studies, Seoul National University, South Korea} \hfill {\em 2018}
\end{rSection}
\bigskip

%----------------------------------------------------------------------------------------
%	Research Interests
%----------------------------------------------------------------------------------------
\begin{rSection}{Research Interests}
%--copy and paste this region  if you need more--
Macroeconomics, Public Finance, Demographic Transition
\end{rSection}
\bigskip
%--------------------------------------------------------------------------------
%    PROJECTS
%-----------------------------------------------------------------------------------------------
\begin{rSection}{Work in Progress}
\medskip
{\bf Public Debt and Investment in Aging Developed Economies} \emph{(Job Market Paper)} \\
\begin{adjustbox}{minipage=0.95\textwidth, right}
	\vspace{0.3em} {\emph{Abstract:} This paper investigates government foreign investment as a strategy to manage public debt in aging developed economies. Using a quantitative overlapping generations model with incomplete markets, I show that it is possible for a government to sustain permanent fiscal deficits and maintain a finite amount of debt even when the domestic interest rate exceeds the economic growth rate. This is achievable if the government invests a sufficiently large share of its debt in foreign assets or if the return on foreign investment is high enough to offset the unfavorable domestic interest rate and growth rate differential. It is also found that government foreign investment can help stabilize the debt-to-GDP ratio in the long run as long as the return on foreign assets is higher than the government's borrowing cost, without relying on an active fiscal rule targeting a debt-to-GDP ratio. Calibrated to the Japanese economy from 1995 to 2020, the model explains 72.22\% of the increase in Japan's debt-to-GDP ratio during this period, with the government's foreign investment strategy accounting for 55\% of the debt accumulation. A counterfactual analysis indicates that while this external investment initially increases debt and crowds out domestic capital in the short run, it ultimately leads to a lower debt-to-GDP ratio, a larger capital stock, and higher wages in the long run. A welfare analysis follows and shows that Japanese government's foreign investment strategy is welfare-improving in terms of utilitarian social welfare but not Pareto-improving, and a self-financed lump-sum compensation plan that makes Pareto-improving redistribution is not feasible. This study provides an ideal laboratory for studying fiscal issues in aging economies and valuable insights for other advanced economies facing similar demographic and fiscal challenges.}
\end{adjustbox}

\newpage
\medskip
{\bf Granular Search in Monopsonistic Labor Market} with \emph{Sean McCrary} \\
\begin{adjustbox}{minipage=0.95\textwidth, right}
	\vspace{0.3em} {\emph{Abstract:} We develop a model of labor market equilibrium with the granular search protocol, à la Jarosch, Nimczik and Sorkin(2024), where the firm size distribution is endogenously determined. We introduce a simple static model to show that the monopsonist's effective Nash bargaining power is endogenously determined by the firm's relative size. Monopsonistic firms choose the optimal vacancy postings taking into account the wage determination, ending up with overposting vacancies to suppress wages. We then extend the model to a  dynamic model and solve its stationary equilibrium. The dynamic model delivers a rich set of policy implications, including the effect of minimum wage policy on the wage distribution and the effect of competition policy on the labor market equilibrium. Specifically, we show that the increase in wages upon the entry of a new firm is more amplified in the presence of granular search protocol since the new firm's entry reduces the incumbent firm's effective bargaining power as well as their employment size.}
\end{adjustbox}\\

\medskip
{\bf Sovereign Partial Default in Continuous Time} with \emph{Gabriel Mihalache} \\
	 \begin{adjustbox}{minipage=0.95\textwidth, right}
		\vspace{0.3em} {\emph{Abstract:} We formulate and solve a tractable, continuous time version of the sovereign \emph{partial default} model of Arellano, Mateos-Planas and Ríos-Rull (2023). We compute our model using both traditional continuous time methods and, with an eye towards larger state space applications, on a deep neural network. We show that our formulation allows for a tight characterization of debt and default dynamics, as well as the length and severity of crisis events.}
	\end{adjustbox}\\

\medskip
{\bf Strategic Demand for Inventors} \\
	 \begin{adjustbox}{minipage=0.95\textwidth, right}
		\vspace{0.3em} {\emph{Abstract:}This paper develops a simple Schumpeterian growth model where firm-level strategic demand for inventors can be described.  All firms should produce the latest invention to be the monopolist in output market, and inventor is the only input of innovation.  In the model, innovating firms can determine their demands for inventors to strategically deter their competitors' innovation.  Using a tractable model with Stackelberg competition in inventor market,  I study the effect frontier firm's strategic demands on the aggregate growth.  Compared to the non-strategic model,  frontier firms' strategic hiring decisions can worse off the aggregate growth and the top income inequality.  The mechanism is intensified when fixed cost of R\&D is high.} \\
		\emph{Presented at: Midwest Macro Spring 2023 (Clemson)}
	\end{adjustbox}\\

%--copy and paste this region  if you need more--
\end{rSection}
\bigskip

%----------------------------------------------------------------------------------------
%    TEACHING
%----------------------------------------------------------------------------------------
\begin{rSection}{Teaching}
	{\bf Independent Instructor} \\
	\begin{adjustbox}{minipage=0.95\textwidth, right}
		\vspace{0.3em} 
		- Intermediate Macroeconomics Theory (2022 Fall, 2023 Fall) \\
		- Principles of Microeconomics (2022 Summer) 
	\end{adjustbox}

	\medskip
	{\bf Recitation Leader} \\
	\begin{adjustbox}{minipage=0.95\textwidth, right}
		\vspace{0.3em} 
		- Macroeconomics Theory 2B (Ph.D., Prof. Lam) (2025 Spring) \\
		- Macroeconomics Theory 1B (Ph.D., Prof. Thomas) (2024 Fall) \\
		- Principles of Microeconomics (Undergraduate, Dr. Buser) (2024 Spring)
	\end{adjustbox}

	\medskip
	{\bf Teaching Award} \\
	\begin{adjustbox}{minipage=0.95\textwidth, right}
		\vspace{0.3em} 
		- Departmental Graduate Associate Teaching Awards (2025)
	\end{adjustbox}
\end{rSection}


%----------------------------------------------------------------------------------------
%    ACTIVITIES
%----------------------------------------------------------------------------------------
\newpage
\begin{rSection}{Conferences}
{\bf 2023} Midwest Macro Spring (Clemson) \\
{\bf 2022} Princeton Initiative: Macro, Money and Finance (Princeton)
\end{rSection}
\bigskip


%----------------------------------------------------------------------------------------
%	SKILLS SECTION
%----------------------------------------------------------------------------------------
\begin{rSection}{Computation and Data}
Julia, Python, Stata
\end{rSection}
\bigskip

%----------------------------------------------------------------------------------------
%	References
%----------------------------------------------------------------------------------------
\begin{rSection}{References} 
\begin{minipage}{6cm}
	{\bf Gabriel Mihalache} \emph{(advisor)} \\
	Assistant Professor of Economics \\
	Department of Economics \\
	The Ohio State University \\
	mihalache.2@osu.edu
\end{minipage}
\begin{minipage}{6cm}
	{\bf Pok-Sang Lam} \\
	Associate Professor of Economics \\
	Department of Economics \\
	The Ohio State University \\
	lam.1@osu.edu
\end{minipage}
\begin{minipage}{6cm}
	{\bf Sean McCrary} \\
	Assistant Professor of Economics \\
	Department of Economics \\
	The Ohio State University \\
	mccrary.65@osu.edu
\end{minipage}

\bigskip
\begin{minipage}{6cm}
	{\bf Julia K. Thomas} \\
	Professor of Economics \\
	Department of Economics \\
	The Ohio State University \\
	thomas.2108@osu.edu
\end{minipage}

\end{rSection}

\end{document}

